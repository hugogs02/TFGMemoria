\pagestyle{plain}
\chapter*{Resumo}
O cambio climático ten na presenza de múltiples axentes contaminantes na atmosfera unha das súas princpais causas. Actualmente, ademais, dispoñemos de múltiples satélites que nos permiten monitorizar
os niveis destes parámetros de forma remota. Porén, analizar multitude de arquivos e datos, para diversos parámetros e días, pode ser unha tarefa tediosa. É por iso que se propón este traballo, que
ten como obxectivo deseñar un sistema para monitorizar a variabilidade e tendencia de diferentes parámetros obtidos a través de datos de satélite. Mediante a descarga destes datos e a súa transformación
e agregación, poderemos reducir significativamente a cantidade de arquivos que será preciso analizar, o cal facilita o seguimento da contaminación por parte dos usuarios finais. O desenvolvemento deste
sistema farase de forma que se poida acceder a el facilmente, e tendo en conta tamén a interoperabilidade do mesmo.