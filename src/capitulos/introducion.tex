\chapter{Introdución}\label{introducion}
\hyphenation{In-te-re-se}

O cambio climático é un asunto que, actualmente, está á orde do día. O aumento das temperaturas a nivel global é un asunto que preocupa non só ós
científicos, senón tamén á poboación en xeral. As causas do cambio climático son moitas, pero podemos estar de acordo en que unha gran parte é
debida á presenza de múltiples gases ou substancias químicas que afectan á atmosfera.

A presenza destes axentes contaminantes tamén afecta en gran medida ós seres humanos. Moitas enfermidades presentes na sociedade de hoxe en día son debidas
a un exceso de contaminación atmosférica. Doenzas respiratorias, coma a asma, ou alerxias son problema de saúde que están cada vez máis presentes, e que afectan
negativamente á calidade de vida das persoas.

Por iso, monitorizar a evolución destes axentes é algo crucial, sobre todo en entornos especialmente críticos, como poden ser as cidades, nas cales a conxunción
dunha gran densidade de poboación e de movementos en áreas, polo xeral, pequenas, fai que sexa crucial coñecer cal é o nivel destes axentes. Desta forma, as
autoridades poden adoptar medidas, tanto de saúde pública (como poden ser plans de prevencións de diferentes enfermidades), coma de proteción do medio ambiente
(limitando, por exemplo, a circulación de vehículos), co obxectivo de evitar que se causen danos ó entorno en que vivimos e reducir a mortalidade debida a
estas enfermidades.

Así, dende a Axencia Espacial Europea (ESA) lanzouse, no ano 2017, un satélite de observación terreste nomeado Sentinel-5 Precursor (ou Sentinel-5P).
Dito satélite, que se enmarca dentro do Programa COPERNICUS, emprega un instrumento denominado espectrómetro, co obxectivo de monitorizar a contaminación
atmosférica. Desta forma, TROPOMI proporciona observacións de forma diaria de gases atmosféricos que son chave á hora de monitorizar a calidade do aire
e realizar predicións sobre a evolución da mesma, coma poden ser o metano ($CH_{4}$), monóxido de carbono ($CO$), formaldehído ($HCHO$), dióxido de nitróxeno
($NO_{2}$), ozono ($O_{3}$), aerosol ou dióxido de xofre ($SO_{2}$).

Este satélite proporciona datos a dous niveis distintos: L1, con datos crus; e L2, con variables xeofísicas derivadas dos datos de nivel 1. Existe outro nivel,
o Nivel 3 (L3), no cal as variables están mapeadas en escalas regulares. Será este o nivel que nos interese alcanzar nos nosos datos, para poder obter con eles
un agregado mensual que nos permita ver a media ou crear climatoloxías para poder observar anomalías nos mesmos.

O obxectivo deste traballo é, por tanto, empregar os datos que nos proporciona este satélite para, mediante a realización de diferentes operacións sobre
os mesmos, poder facer un seguimento da evolución ó longo do tempo da contaminación en diversas cidades de España. Decidiuse escoller este ámbito debido
a que, ó tratarse dun área relativamente reducida, o procesamento dos datos será máis sinxelo, posto que será preciso procesar menos arquivos de menor tamaño.


\section{Obxectivos do traballo}\label{obxectivos}
Os obxectivos deste traballo de fin de grao son os seguintes:
\begin{enumerate}
    \item Mapear a contaminación atmosférica a partir de datos de sensores remotos, con especial relevancia en entornos urbanos
    \item Monitorizar a variabilidade e tendencia dos diferentes parámetros satelitais relacionados coa contaminación do aire.
    \item Desenvolver un visor online que permita representar de forma dinámica a información.
    \item Implementar un sistema interoperable de distribución dos datos e produtos.
\end{enumerate}

\section{Estrutura da memoria}\label{estrutura}
A presente memoria estrutúrase do seguinte xeito:
\begin{itemize}
    \item \textbf{Capítulo 2:} Especificación de requisitos. Este capítulo describe detalladamente os requisitos e criterios que debe cumprir o proxecto.
    \item \textbf{Capítulo 3:} Deseño. Aquí explicaremos todo o relativo ó proceso de deseño do software ata acadar a súa versión final. Tamén se comenta a
    metodoloxía empregada para o seu desenvolvemento, así como as diferentes tecnoloxías das que se fixo uso no mesmo.
    \item \textbf{Capítulo 4:} Probas. Esta sección trata sobre as probas realizadas, así como os resultados das mesmas.
    \item \textbf{Capítulo 5:} Conclusións e posibles ampliacións. Neste último capítulo preséntanse as conclusións obtidas após realizar o proxecto,
    así como propostas de melloras sobre o mesmo ou posibles ampliacións futuras.
\end{itemize}

\section{Descrición do sistema}\label{descricion}
Como se explicou nos \hyperref[sec:obxectivos]{obxectivos do traballo}, o sistema que deseñemos terá como finalidade poder monitorizar a variabilidade e tendencia
de diferentes parámetros, obtidos a través de datos de satélite, os cales se empregan para medir a contaminación do aire.

Será preciso, por tanto, deseñar un módulo  mediante o cal poidamos descargar os datos de TROPOMI, obtidos a través da ESA, para poder traballar con eles, realizando
as transformacións anteriormente mencionadas. Así, empregando as APIs proporcionadas, poderemos deseñar unha función que busque un determinado produto (por exemplo,
$NO_{2}$) no catálogo de COPERNICUS, filtrando os resultados para centrarnos nun área de interese (neste caso, a Península Ibérica) e datas concretas. Construíndo unha
query, e empregando a librería \texttt{requests}, descagaremos os arquivos para, posteriormente, transformalos. Ademais, os arquivos veñen en formato \texttt{.zip}, polo que será
preciso, neste módulo, descomprimilos para poder obter os arquivos de formato \texttt{.nc} cos que debemos traballar. Este módulo será desenvolvido na linguaxe Python.

Por outra parte, deseñaremos outro módulo, tamén en Python, no cal poidamos traballar cos arquivos descargados. Os datos que descargamos son de nivel 2, polo que temos que realizar
operacións cos mesmos para transformalos en datos agregados por mes. Para isto, será preciso empregar \texttt{HARP}, unha ferramenta que permite ler e procesar datos de satélite,
permitindo a inxesta de arquivos e o traballo cos mesmos. Así, importaremos tódolos arquivos descargados e realizaremos as transformacións anteriormente citadas, para posteriomente
exportar o resultado nun arquivo, tamén de formato \texttt{.nc}, nun directorio específico.

Os datos transformados almacenaranse nun servidor ERDDAP, que empregaremos como middleware. Este servidor proporciona unha forma sinxela de acceder a datasets científicos en formatos
de arquivos comúns, facilitando a xeración de gráficos e mapas. Unifica diferentes tipos de servidores con diferentes estruturas e devolve os datos nun formato de arquivo común, así
como estandariza as datas entre os mesmos. Ademais, é posible configurar un servidor ERDDAP propio e empregar os nosos propios datos. Esta será a opción que elixamos, xa que nos permite
empregar os datos que nós mesmos transformamos.

Será preciso establecer, no servidor onde corra a aplicación, unha tarefa automática, para que, segundo se publiquen datos novos, estes estean dispoñibles na nosa aplicación. Esta tarefa
deberá executarse tódolos meses, tendo en conta que os datos das medicións de satélite tardan uns días en estaren dispoñibles, e deberá descargar os datos dos parámetros que empregue a
nosa aplicación, transformalos e almacenalos no directorio establecido.

Por último, deseñaremos unha interface web que permitirá consultar a evolución de cada un dos diferentes parámetros (escollendo un deles) ó longo do tempo.

Todo isto implementarémolo nunha máquina de AWS, de forma que o sistema se administrará dende a mesma. Así, poderemos acceder á interface web de forma sinxela, a través de internet,
e toda a xestión da aplicación estará tamén centralizada. Nesta máquina almacenaremos os datos transformados que obteñamos do módulo de Python, e nela teremos tamén o servidor ERDDAP
co cal acceder a ditos datos. Accedendo á máquina poderemos visualizar a interface web que mencionamos.
