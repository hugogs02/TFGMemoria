\chapter{Especificación de Requisitos}

%Especificación dos requisitos máis relevantes do Sistema, xunto coa información que este debe almacenar e as interfaces
%con outros Sistemas, sexan hardware ou software, e outros requisitos (rendemento, seguridade, etc.).

Neste capítulo trataremos en profundidade os principais requisitos que debe cumplir o sistema e o traballo; isto é, as funcionalidades que se require que teña o sistema.
Especificaremos os usuarios que empregarán o sistema, as funcionalidades que se lle atribúen a cada un e os compoñentes do sistema, así como os requisitos funcionais e non
funcionais do mesmo.

\section{Identificación dos usuarios}\label{usuarios}
O obxectivo deste proxecto é proporcionar información de diferentes parámetros relacionados coa contaminación atmosférica, co obexectivo de poder realizar un seguimento de
como evolucionou ó longo do tempo a contaminación en diferentes puntos da Península Ibérica. Temos, por tanto, un perfil de usuario claro: aquel que consultará os datos, ou
usuario final [U1]. Por outra parte, identificaremos outro usuario, administrador [U2] que será o encargado de realizar o mantemento de todo o sistema, e que se encargará de arranxar os
distintos problemas que poidan surxir co uso do mesmo.

\begin{itemize}
    \item O \textbf{usuario final ([U1])} é aquela persona á que vai dirixida principalmente o sistema, e que ten interese en poder monitorizar a evolución da contaminación por
    diferentes parámetros ó longo do tempo. Trátase dun perfil de persoa variada, dende un científico, un traballador público que queira elaborar plans para a redución da contaminación
    nas cidades, ou calquera cidadán con interese polo tema. Porén, as tarefas que se lle atribúen son sinxelas:
    \begin{enumerate}
        \item \textbf{Consultar o nivel de contaminación.} Este usuario consultará, na interface web do servidor en que se desenvolva o sistema, o valor dun parámetro en concreto, de entre
        tódolos que estean dispoñibles, para un mes concreto.
    \end{enumerate}
    \item O \textbf{administrador do sistema [U2]} é a persoa que se encargará do mantemento do mesmo, monitorizando o funcionamento do servidor para que, en caso de que se produza algún
    erro, poder solventalo. O rol de administrador do sistema asumiráo o alumno encargado de desenvolver este traballo. As súas principais tarefas serán:
    \begin{enumerate}
        \item \textbf{Configurar o sistema}. Este usuario será o encargado de realizar a configuración inicial do sistema, implementando os diferentes módulos que o compoñen
        e descargando e procesando datos durante un periodo de tempo o suficientemente longo, que permita a correcta utilización do sistema.
        \item \textbf{Monitorear o sistema}. Como administrador, deberá encargarse de monitorizar o correcto funcionamento do sistema, vixiando os posibles problemas que poidan xurdir no
        mesmo. Especialmente, deberá facer énfase en controlar que, cada mes, se descarguen e procesen correctamente os arquivos correspondentes, para asegurar a máxima calidade nos datos
        que consulte o usuario final.
    \end{enumerate}
\end{itemize}


\section{Requisitos funcionais}
\subsection{Obter datos de satélite da ESA [RF1]}\label{rf1}
A aplicación que se desenvolva debe ser capaz de obter, de forma automática, os datos de satélite dispoñibles no dataspace de COPERNICUS,
na web da Axencia Espacial Europea. Isto farase empregando unha API de entre aquelas que ofrece a propia ESA. Estes datos deberán limitarse
a unha área xeográfica de interese, que no noso caso será a Península Ibérica, e a un rango de datas determinados.

\subsection{Transformar e agregar datos de satélite [RF2]}\label{rf2}
Os datos que se descarguen dende a web da ESA deberán ser transformados, de forma que serán agregados mensualmente para poder facer un
seguemento da evolución da contaminación. Os arquivos resultantes deberán manter unicamente a información necesaria.

\subsection{Implementar un servidor ERDDAP local [RF3]}\label{rf3}
No servidor onde se implemente este sistema, implementarase un servidor ERDDAP local, de forma que se facilite o acceso ós datos. Deberase
configurar un \textit{dataset} por cada un dos parámetros, e almacenar nel os diferentes arquivos.

%\subsection{Obter datos de satélite de TROPOMI [RF4]}\label{rf4}

\section{Requisitos do produto}
\subsection{A aplicación deberá ser rápida e eficiente [RNF1]}\label{rnf1}
A aplicación deberá ser capaz de, nun tempo razoable dende que estean dispoñibles, descargar e procesar os datos de satélite, de forma que
os usuarios poidan acceder a estos datos o antes posible.

\subsection{Eficiencia de almacenamento [RNF2]}\label{rnf2}
A aplicación debe ser eficiente en canto ó uso do almacenamento se refire. Tendo en conta que, aproximadamente, cada mes se descagarán en torno
a 50GB de novos datos por cada produto que se procese, e que após o procesamento teremos datos cun tamaño resultante darredor de 500MB por
parámetro, é crucial que a aplicación almacene só os datos estritamente necesarios.

\section{Requisitos de rendemento}
\subsection{Tempo de resposta razoable [RNF3]}\label{rnf3}
A aplicación debe ter un tempo de resposta razoable, de forma que os usuarios non deban agardar moito tempo para teren os resultados que desexan.