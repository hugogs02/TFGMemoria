\chapter{Manuais técnicos}\label{manuais}

%En función do tipo de Traballo e metodoloxía empregada, o contido poderase dividir en varios documentos. En todo caso,
%neles incluirase toda a información precisa para aquelas persoas que se vaian encargar do desenvolvemento e/ou modificación
%do Sistema (por exemplo código fonte, recursos necesarios, operacións necesarias para modificacións e probas, posibles
%problemas, etc.). O código fonte poderase entregar en soporte informático en formatos PDF ou postscript.
Este manual técnico explica como realizar a instalación do sistema e o seu mantemento por parte dos administradores.
\section{Instalación}\label{instalacion}
No repositorio de GitHub \url{https://github.com/hugogs02/TROPOMIVisor} dispoñemos de todo o código fonte da aplicación, tanto do backend de Python, coma da interface e os arquivos de configuración do
servidor ERDDAP. É preciso recalcar que, no caso de querer instalar o sistema no ordenador propio, este deberá empregar o sistema operativo Ubuntu. Para a súa instalación, é preciso seguir os
seguintes pasos:
\begin{enumerate}
    \item \textbf{Instalar Python}. O sistema require \textbf{Python 3.11.6}, o cal se instalará mediante Anaconda, para facilitar tamén a xestión e instalación de paquetes. Para instalar Anaconda
    , debemos descargar o instalador dispoñible en \url{https://www.anaconda.com/download#downloads} . Deberase dar permiso de execución (\texttt{chmod o+rx [filename]}). Será necesario seguir os
    pasos que indica o instalador, entre os que se inclúen elixir o directorio de instalación, engadilo ó PATH e inicializar a instalación co comando \texttt{conda init}. A continuación, 
    instalaremos os seguintes paquetes:
    \begin{itemize}
        \item calendar (vén preinstalado con Python).
        \item datetime (versión 5.4).
        \item dateutil.parser (versión 2.8.2).
        \item glob2 (versión 0.7).
        \item harp (versión 1.17). Instalado con conda-forge.
        \item os (vén preinstalado con Python).
        \item pandas (versión 2.1.4).
        \item requests (versión 2.31.0).
        \item shutil (vén preinstalado con Python).
        \item sys (vén preinstalado con Python).
        \item time (vén preinstalado con Python).
        \item zipfile36 (versión 0.1.3).
    \end{itemize}
    \item \textbf{Instalar Java}. A instalación de Java debe ser da versión 17 (ou superior). Instalar Java en Ubuntu é sinxelo, simplemente se debe executar o comando \texttt{apt install openjdk-17-jdk
    openjdk-17-jre} .
    \item \textbf{Configurar Tomcat}. Para configurar correctamente tomcat, débese crear un usuario tomcat (con \texttt{sudo useradd -m -d /opt/tomcat -U -s /bin/false tomcat}), e descargar o
    arquivo de Tomcat (versión 10) dende \url{https://tomcat.apache.org/download-10.cgi} . Este deberá ser descomprimido no directorio \textit{opt/tomcat} Accederase ó arquivo \texttt{/opt/tomcat/conf/tomcat-users.xml}
    para configurar os usuarios Manager e Host Manager. A continuación, crearase un servizo para Tomcat, mediante o comando \texttt{sudo nano \break/etc/systemd/system/tomcat.service}, engadindo as
    seguintes liñas:
    \lstset{numbers=left,breaklines=true}
    \begin{lstlisting}
    [Unit]
    Description=Tomcat
    After=network.target

    [Service]
    Type=forking

    User=tomcat
    Group=tomcat

    Environment="JAVA_HOME=/usr/lib/jvm/java-1.11.0-openjdk-amd64"
    Environment="JAVA_OPTS=-Djava.security.egd=file:///dev/urandom"
    Environment="CATALINA_BASE=/opt/tomcat"
    Environment="CATALINA_HOME=/opt/tomcat"
    Environment="CATALINA_PID=/opt/tomcat/temp/tomcat.pid"
    Environment="CATALINA_OPTS=-Xms512M -Xmx1024M -server -XX:+UseParallelGC"

    ExecStart=/opt/tomcat/bin/startup.sh
    ExecStop=/opt/tomcat/bin/shutdown.sh

    RestartSec=10
    Restart=always

    [Install]
    WantedBy=multi-user.target
    \end{lstlisting}
    A continuación, recargarase o \textit{systemctl} con \texttt{sudo systemctl daemon-reload} e arrancarase o servizo de Tomcat con \texttt{sudo systemctl start tomcat}. Configurarase Tomcat para
    que arranque co sistema, mediante o comando \texttt{sudo systemctl enable tomcat}. Por último, executarase \texttt{sudo ufw allow 8080} para permitir tráfico polo porto 8080.
    \item \textbf{Executar os scripts de Python para obtención de datos}. Empregando git, podemos descagar o código dispoñible no repositorio que indicamos anteriormente. Co código descargado, 
    podemos executar o script principal para descargar os datos. O formato de execución será: \texttt{python app.py produto data\_inicio data\_fin ruta\_descarga ruta\_procesado}. ``produto`` deberá
    ser un entre os dispoñibles (L2\_\_CO\_\_\_\_, L2\_\_NO2\_\_\_, L2\_\_O3\_\_\_\_, L2\_\_SO2\_\_\_), e a data deberá ter o formato AAAA-MM-DD
    \item \textbf{Configurar o servidor ERDDAP}. Para configurar o servidor ERDDAP, deberemos seguir as instrucións de \url{https://coastwatch.pfeg.noaa.gov/erddap/download/setup.html}, empregando
    os arquivos de setup.xml e datasets.xml que se descargaron do repositorio de GitHub.
    \item \textbf{Configurar a interface web}. Para a interface web, crearase unha aplicación nova de Tomcat. Isto farase creando unha nova carpeta no directorio \textit{/opt/tomcat/webapps/}, e
    copiando á mesma o arquivo HTML que se descargou do repositorio.
\end{enumerate}

Unha vez seguidos todos estes pasos, o sistema estará listo para o seu uso.

\section{Mantemento}\label{mantemento}
O mantemento do sistema é sinxelo, pois o administrador só precisa consultar o log de erros o día que se indicou no \textit{job} de cron para executar a actualización dos datos, e comprobar que
funcionou correctamente. En caso de que se rexistre algunha mensaxe de erro relativa á descarga dos arquivos, se esta se dá nun ou dous deles, pode ignorarse, posto que non afectara ó resultado. En
caso de que se produzan máis erros, será preciso executar o script principal manualmente, para realizar a descarga de datos. O mesmo será preciso en caso de ter algún erro coa importación e
transformación dos datos.

Tamén poden ocurrir erros ou problemas co servidor ERDDAP. Neste caso, o mellor é reiniciar Tomcat, empregando para iso os scripts \texttt{shutdown.sh} e \texttt{startup.sh}, ubicados en
\textit{/opt/tomcat/bin/} .