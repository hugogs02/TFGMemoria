\chapter{Conclusións e posibles ampliacións}
O obxectivo principal deste traballo era deseñar un sistema para a monitorización da variabilidade e tendencia de diferentes parámetros empregados para medir a contaminación do aire, con datos
obtidos mediante medicións de satélite. Así, buscábase aproveitar os datos dispoñibles na web de COPERNICUS para traballar con eles, creando agregados mensuais, de forma que os usuarios os puideran
empregar para seguir a evolución da contaminación atmosférica.

Este sistema ofrece a posibilidade de consultar, para a Península Ibérica, os valores de catro parámetros diferentes ó longo do ano 2023. Estes parámetros son o monóxido de carbono ($CO$), o ozono
($O_3$), o dióxido de nitróxeno ($NO_2$) e o dióxido de xofre ($SO_2$). Mediante unha interface sinxela, podemos visualizar nun mapa o valor destes parámetros, que se mapearon a unha cuadrícula
regular que abarca toda a área de interese.

O usuario ten, desta forma, fácil acceso ós datos de satélite, podendo visualizar os mesmos dunha forma sinxela. Posto que os arquivos individuais de cada un dos parámetros teñen un gran tamaño,
mediante a agregación dos mesmos de forma mensual podemos reducir drasticamente as necesidades de almacenamento, o cal resulta realmente beneficioso tendo en conta que, para un ano, poderiamos
estar falando de cantidades en torno a 300GB de datos para cada parámetro por ano, mentres que o noso sistema precisa menos de 1GB por parámetro.

Desta forma, podemos considerar que se satisfixeron os catro obxectivos que se definiron en \ref{obxectivos}. Logrouse mapear a contaminación atmosférica mediante os datos de sensores remotos,
mediante a transformación dos datos dispoñibles no dataspace COPERNICUS da ESA e a súa transformación a nivel L3. Conseguimos monitorizar a variabilidade e tendencia dos parámetros relacionados coa
contaminación do aire, tanto de forma xeográfica, visualizándoos nun mapa, como a través de gráficos de liñas. En relación con isto, tamén desenvolvimos un sinxelo visor online, para representar a
información de forma gráfica e dinámica. Por último, implementouse un sistema interoperable para distribuír os datos, posto que o propio servidor ERDDAP permite compartir e distribuír os datos con
outros sistemas, dunha forma uniforme.

Para rematar, propóñense unha serie de melloras que se poderían aplicar ó sistema para facelo máis completo e adaptable ás necesidades dos usuarios:
\begin{enumerate}
    \item \textbf{Maior dispoñibilidade de datos}. Os parámetros dispoñibles no dataspace de COPERNICUS inclúen, ademais dos analizados neste traballo, o metano ($CH_4$), formaldehído ($HCHO$), a
    fracción de nubes ou o índice UV. Estes parámetros tiveron que descartarse do proxecto por falta de tempo, polo que podería ser interesante incluílos en ampliacións futuras. Asimesmo, os datos
    están dispoñibles dende mediados do ano 2018, polo que tamén sería factible ampliar a dispoñibilidade temporal dos mesmos, que actualmente é dende xaneiro de 2023 ó presente. Tamén se podería
    ampliar a área de análise dos datos a todo o continente europeo, ou incluso a todo o mundo.
    \item \textbf{Capacidade de automatización}. Podería ser útil integrar utilidades de automatización no sistema, de forma que os usuarios puideran recibir, de forma automática, algúns gráficos
    que eles mesmos elixan segundo teñamos novos produtos dispoñibles, para facilitármoslles a análise dos mesmos.
    \item \textbf{Mellora da experiencia e interface de usuario}. Unha forma de mellora da interface de usuario sería crear unha interface de JavaScript empregando leaflet. Isto non puido levarse a
    cabo neste proxecto, pero podería ser práctico en caso de aumentar tamén a dispoñibilidade de datos, xa que nos permite unha maior interacción cos mapas. Tamén poderían implementarse
    visualizacións en forma de gráficos 3D.
    \item \textbf{Creación dun \textit{dashboard} para visualizar os parámetros en distintas zonas da Península}. Proponse a creación dun taboleiro que permita, a partir dos datos dos que se dispón,
    centrarse en diferentes áreas urbanas da Península (por exemplo, a área metropolitana de Madrid ou a de Barcelona), para analizar máis en profundidade o valor e a evolución dos diferentes par
    ámetros. Tamén se podería introducir, neste taboleiro, un apartado mediante o cal visualizar as zonas quentes; é dicir, aquelas zonas con valores especialmente altos de diferentes parámetros
    nun mes en concreto, ou de forma prolongada no tempo.
    \item \textbf{Integración con Machine Learning}. Como un campo que está actualmente en auxe, o machine learning resulta especialmente relevante no ámbito da monitorización ambiental, xa que nos
    permitiría, a partir dos datos dos que dispoñemos, identificar patróns nos mesmos co obxectivo de elaborar predicións. Desta forma, poderíamos crear, por exemplo, agregacións semanais no canto
    de mensuais, o que nos permitiría dispoñer de información máis pormenorizada e realizar mellores análises. Isto podería empregars en colaboración con diferentes cidades, para saberen cando
    activar, por exemplo, os seus plans anticontaminación.
    \item \textbf{Análises estatísticos}. Por último, proponse a incorporación de ferramentas estatísticas para mellorar a experiencia dos usuarios á hora de analizar os datos. Por exemplo, poder
    ían implemantarse algoritmos para a detección de anomalías, ou para realizar directamente análises temporais e seguementos dos cambios dos diferentes parámetros ó longo do tempo.
\end{enumerate}
